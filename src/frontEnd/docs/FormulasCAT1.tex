\documentclass{article}
\usepackage{amsmath}

\begin{document}

\section*{Fórmulas para Circuitos Exame CAT 1}



\begin{itemize}
    \item \textbf{Frequência de Ressonância (\( f_r \))}:
    \[ f_r = \frac{1}{{2 \pi \sqrt{LC}}} \]
    Onde:
    \begin{itemize}
        \item \( f_r \) é a frequência de ressonância.
        \item \( L \) é a indutância do circuito.
        \item \( C \) é a capacitância do circuito.
    \end{itemize}
    
    \item \textbf{Reatância Capacitiva (\( X_C \))}:
    \[ X_C = \frac{1}{{2 \pi f C}} \]
    Onde:
    \begin{itemize}
        \item \( X_C \) é a reatância capacitiva.
        \item \( f \) é a frequência.
        \item \( C \) é a capacitância.
    \end{itemize}
    
    \item \textbf{Reatância Indutiva (\( X_L \))}:
    \[ X_L = 2 \pi f L \]
    Onde:
    \begin{itemize}
        \item \( X_L \) é a reatância indutiva.
        \item \( f \) é a frequência.
        \item \( L \) é a indutância.
    \end{itemize}
    
    \item \textbf{Largura de Banda (\( BW \))}:
    \[ BW = \frac{{\text{Frequência de Ressonância}}}{{Q}} \]
    Onde:
    \begin{itemize}
        \item \( BW \) é a largura de banda Hz.
        \item \( Q \) é o fator de qualidade do circuito.
        \item \text{Frequência de Ressonância} é a frequência em que a resposta em frequência é máxima.
    \end{itemize}
    \section*{Fórmulas Adicionais para Circuitos}

\subsection*{Capacitores em Série e Paralelo}

Para capacitores em série, a capacitância total (\(C_{\text{total}}\)) é dada por:

\[
\frac{1}{{C_{\text{total}}}} = \frac{1}{{C_1}} + \frac{1}{{C_2}} + \ldots + \frac{1}{{C_n}}
\]

Para capacitores em paralelo, a capacitância total é a soma das capacitâncias individuais:

\[
C_{\text{total}} = C_1 + C_2 + \ldots + C_n
\]

\subsection*{Bobinas em Série e Paralelo}

Para bobinas em série, a indutância total (\(L_{\text{total}}\)) é a soma das indutâncias individuais:

\[
L_{\text{total}} = L_1 + L_2 + \ldots + L_n
\]

Para bobinas em paralelo, a indutância total é dada por:

\[
\frac{1}{{L_{\text{total}}}} = \frac{1}{{L_1}} + \frac{1}{{L_2}} + \ldots + \frac{1}{{L_n}}
\]

\subsection*{Transformador: Relação de Tensão e Corrente}

A relação entre a tensão no secundário (\(V_s\)) e no primário (\(V_p\)) de um transformador é dada por:

\[
\frac{{V_s}}{{V_p}} = \frac{{N_s}}{{N_p}}
\]

onde \(N_s\) é o número de espiras no secundário e \(N_p\) é o número de espiras no primário.

Similarmente, a relação entre a corrente no secundário (\(I_s\)) e no primário (\(I_p\)) é dada por:

\[
\frac{{I_p}}{{I_s}} = \frac{{N_s}}{{N_p}}
\]

\subsection*{Fórmula de Atenuação de Sinal}

Para calcular a atenuação do sinal em decibéis, utilizamos a seguinte fórmula:
\[
S = 20 \times \log_{10}\left(\frac{E_i}{E_t}\right)
\]

onde $S$ representa a atenuação do sinal em decibéis (dB), $E_i$ é a intensidade do sinal de entrada e $E_t$ é a intensidade do sinal transmitido. Esta fórmula é fundamental para entender como diferentes materiais influenciam na propagação de ondas eletromagnéticas.

\section*{Fórmulas de Fator de Velocidade e Comprimento de Onda}

O fator de velocidade de um cabo coaxial é dado pela fórmula:
\[
v_p = v_c \times F_v
\]
onde:
\begin{itemize}
    \item \( v_p \) é a velocidade de propagação do sinal no cabo,
    \item \( v_c \) é a velocidade da luz no vácuo,
    \item \( F_v \) é o fator de velocidade do cabo.
\end{itemize}

\end{itemize}

\end{document}